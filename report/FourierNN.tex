\documentclass[11pt]{article}
%\documentclass[review]{elsarticle}
\usepackage[utf8]{inputenc}

\usepackage{lmodern}
\usepackage{subfig}
\usepackage[english]{babel}
\usepackage{multirow}
%\usepackage{amsmath,amssymb,psfrag, amsthm}
\usepackage{amsmath,amssymb,psfrag}
\usepackage{graphicx}
\usepackage{fullpage}
\usepackage{listings}
\usepackage{paralist}
\usepackage{appendix}
\usepackage{amsfonts}
\usepackage{subfig,color}
\usepackage{comment} 
\usepackage{enumerate}
\usepackage{cancel}
\usepackage{helvet}
\usepackage{epstopdf}
\usepackage{tikz,tikz-cd}
\usepackage{pgfplots,pgfplotstable}
\usepackage{lineno,hyperref}
\usepackage{mathtools}
\usetikzlibrary{pgfplots.groupplots}
\usetikzlibrary{positioning}
\usetikzlibrary[shapes,arrows,trees]
\usetikzlibrary{matrix,decorations.pathmorphing}
\newcommand{\refalg}[1]{Algorithm~\ref{#1}}
\newcommand{\refsec}[1]{Section~\ref{#1}}
\newcommand{\reffig}[1]{Figure~\ref{#1}}
\newcommand{\refsubfig}[1]{Figure~\subref{#1}}
\newcommand{\reftab}[1]{Table~\ref{#1}}
\newcommand{\refeqn}[1]{(\ref{#1})}
%\newcommand{\reflst}[1]{Listing~(\ref{#1})}


%\newcommand{\refalg}[1]{\cref{#1}}
%\newcommand{\refsec}[1]{\cref{#1}}
%\newcommand{\reffig}[1]{\cref{#1}}
%\newcommand{\refsubfig}[1]{\cref{#1}}
%\newcommand{\reftab}[1]{\cref{#1}}
%\newcommand{\refeqn}[1]{\cref{#1}}
%\newcommand{\reflst}[1]{\cref{#1}}
\renewcommand{\d}{\mathrm{d}}
\newcommand{\re}[1]{(\ref{#1})}
\newcommand{\D}{\mathrm{D}}
\newcommand{\bb}[1]{\boldsymbol{#1}}
\def\sgn{\mathop{\rm sgn}}
\newcommand{\remark}[1]{{\color{red} #1}}
% THEOREMS ETC
\newcommand{\vect}[1]{\mathbf{#1} }
\newcommand{\order}[1]{\mathcal{O}(h^{#1})}
\newcommand{\code}[1]{{\tt #1}}
\renewcommand{\familydefault}{\sfdefault}
% -----------------------------------------------------------
% -----------------------------------------------------------
\lstset{backgroundcolor=\color[rgb]{0.92,0.95,1}}
\lstset{rulecolor=\color[rgb]{0.92,0.95,1}}
\lstset{numbers=left}
\lstset{basicstyle=\ttfamily\footnotesize}
\lstset{numberstyle=\footnotesize}

\def\dfdd#1#2{\frac{\partial#1}{\partial#2}}
\newcommand{\erf}{\, \mathrm{erf}}
\newcommand{\erfc}{\, \mathrm{erfc}}
%\renewcommand{\labelenumi}{[\arabic{enumi}]}
\modulolinenumbers[5]

%\normalsize

\begin{document}
%\def\thefigure{\arabic{figure}}
%\def\thetable{\arabic{table}}

\title{A Fourier Neural Network for learning periodic solutions to PDEs}
\author{Marieme Ngom, Oana Marin \footnote{and ..whoever else takes interest}}
\maketitle
\begin{abstract}

\end{abstract}


%\begin{keyword}
%\texttt{elsarticle.cls}\sep \LaTeX\sep Elsevier \sep template
%\MSC[2010] 00-01\sep  99-00
%\end{keyword}


\linenumbers

\section{Introduction}
\textbf(Multiple hidden layers: First one uses sin/cos activation functions, following uses Relu/sigmoid etc?)
\textbf(Make us of convolutions?)
\textbf(Use Differentiation matrices?)


The goal of this work is to learn periodic solutions of partial differential equations using a Fourier based Neural Network. More specifically, we will use a neural network with one hidden layer and with an activation function in the form of Fourier basis. We then compare our results to the one obtained from neural networks using sigmoid and reLU activation functions.

PUT FIGURE OF NN WITH FOURIER BASIS AND SHOW HOW THEY APPROXIMATE ANY L2 PERIODIC  FUNCTION




\bibliographystyle{plain}
\bibliography{bie}
\end{document}

